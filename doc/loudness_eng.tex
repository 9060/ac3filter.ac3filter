\documentclass[12pt]{article}

\usepackage{a4wide}
\usepackage[english]{babel}

%%%%%%%%%%%%%%%%%%%%%%%%%%%%%%%%%%%%%%%%%%%%%%%%%%%%%%%%%%%
% Packages

\usepackage{ifpdf}           % ifpdf macro
\usepackage{url}
\usepackage[usenames]{color} % allows to change text color
\usepackage{tabularx}        % extended tables

%%%%%%%%%%%%%%%%%%%%%%%%%%%%%%%%%%%%%%%%%%%%%%%%%%%%%%%%%%%
% Graphics

\ifpdf
  \usepackage[pdftex]{graphicx}
  \usepackage[pdftex,unicode]{hyperref}
  \DeclareGraphicsExtensions{.pdf,.jpg,.png}
\else
  \usepackage{graphicx}
  \usepackage{hyperref}
  \DeclareGraphicsExtensions{.pdf,.jpg,.png}
\fi

%%%%%%%%%%%%%%%%%%%%%%%%%%%%%%%%%%%%%%%%%%%%%%%%%%%%%%%%%%%
% Links

\newcommand{\urlhome}[1]{\href{http://ac3filter.net}{#1}}
\newcommand{\urlforum}[1]{\href{http://ac3filter.net/forum}{#1}}
\newcommand{\urldonate}[1]{\href{http://order.kagi.com/?6CZJZ}{#1}}
\newcommand{\urldownload}[1]{\href{http://sourceforge.net/project/showfiles.php?group_id=66022&package_id=63808}{#1}}
\newcommand{\filterlinks}
{
  \begin{center}
    \urlhome{AC3Filter home} :: 
    \urldonate{Donate} :: 
    \urldownload{Download} :: 
    \urlforum{Support forum}
  \end{center}
}

%%%%%%%%%%%%%%%%%%%%%%%%%%%%%%%%%%%%%%%%%%%%%%%%%%%%%%%%%%%
% Style of text elements

% name of an AC3Filter option
\newcommand{\optname}[1]{\textbf{``\textsf{#1}''}}

% value of an option
\newcommand{\optval}[1]{\textbf{\textsf{#1}}}

% definition
\newcommand{\definition}[1]{\textbf{\textit{#1}}}

% notes
\newcommand{\note}[1]{\smallskip{\footnotesize #1}\smallskip}

% warnings
\newcommand{\warning}[1]
{
  \begin{center}
    \fbox{\parbox{0.8\hsize}{\textcolor{red}{
      \textbf{Warning!}  #1
    }}}
  \end{center}
}

% tell user to do something
\newcommand{\action}[1]{\textbf{\textsf{#1}}}

% file names
\newcommand{\file}[1]{\textbf{\textsf{#1}}}

% registry keys
\newcommand{\registry}[1]{\textbf{\textsf{#1}}}

% insert an image
\newcommand{\img}[1] % {filename}
{
  \begin{center}
    \includegraphics[width=10cm]{#1}
  \end{center}
}

% insert a figure
\newcommand{\fig}[3] % {filename}{caption}{label}
{
  \begin{center}
    \label{fig:#3}
    \caption{#2}
    \includegraphics[width=10cm]{#1}
  \end{center}
}

% reference
\newcommand{\myref}[1] % {label}
{
  \ref{#1} \nameref{#1}
}

% section
\newcommand{\mysection}[2] % {label}{name}
{
  \section{#2} \label{sec:#1}
}

% subsection
\newcommand{\mysubsection}[2] % {label}{name}
{
  \subsection{#2} \label{ss:#1}
}


\title{AC3Filter\\Loudness and dynamic range}
\author{Alexander Vigovsky}
\date{}
\setcounter{tocdepth}{2}

%%%%%%%%%%%%%%%%%%%%%%%%%%%%%%%%%%%%%%%%%%%%%%%%%%%%%%%%%%%
% BEGIN
%%%%%%%%%%%%%%%%%%%%%%%%%%%%%%%%%%%%%%%%%%%%%%%%%%%%%%%%%%%

\begin{document}

\maketitle
\filterlinks

\section*{}

Audio quality is a very complex concept depending on a huge number of factors and their relationship: type of acoustic system, listening conditions, quality of original record, listener himself, and many others. It is impossible to give some unique recommendations about how to reach good quality. There is no any unique formula and ideal sound: the record prepared for one listening conditions can sound absolutely unacceptable in other environment. However, we will try to present some of basic principles, which can help to understand what to do in each specific case.

\pagebreak
\tableofcontents
\pagebreak

%%%%%%%%%%%%%%%%%%%%%%%%%%%%%%%%%%%%%%%%%%%%%%%%%%%%%%%%%%%
% What is loudness
%%%%%%%%%%%%%%%%%%%%%%%%%%%%%%%%%%%%%%%%%%%%%%%%%%%%%%%%%%%

\section{What is loudness}

What is loudness? This question, which seems to be clear, is hard enough to formalize, since different cases imply absolutely different things.

The sense of loudness is most clear when speaking about \definition{acoustic pressure}, since that is directly perceived by ear.

\begin{quote}
\definition{Acoustic pressure} is an additional pressure, which occurs when the acoustic wave passes through liquid or gaseous medium. When propagating through medium acoustic wave forms some compressions and depressions, which make additional variation of pressure with respect to average value of medium pressure. Thus, the acoustic pressure is a variable part of pressure, i.e. variation of pressure with respect to average value with frequency corresponding to acoustic wave frequency. (Great Soviet Encyclopedia)
\end{quote}

Thus, we can measure any sound - loud sounds produce high pressure, quiet ones produce low pressure. The pressure is measured in Pascals, however, in acoustics it is usually measured in \definition{decibels} (dB) with respect to the \definition{hearing threshold}. By definition, the value of hearing threshold is accepted to pt = 0.00002 Pa = 20 mkPa. The hearing threshold is taken as 0 dB and \definition{loudness} is calculated as l = 20 * log(p / pt), where l [dB] is loudness (in terms of acoustic pressure), p [Pa] is acoustic pressure; pt [Pa] is hearing threshold. At the same time, all audible sounds have positive loudness value; all inaudible sounds (below the hearing threshold) have negative value; loudness change by 6 dB corresponds to the double change in pressure; change by 20 dB corresponds to the tenfold change in pressure. Loudness in terms of acoustic pressure we will call an \definition{absolute loudness}.

Here are some typical loudness values:

\bigskip
\begin{center} \begin{tabular}{|l|l|l|}
\hline                 Sound                 & Loudness  &   Pressure    \\
\hline Hearing threshold                     &    0dB    &    20 mkPa    \\
\hline Rustling of leaves and gentle breeze  &           &               \\
       Wristwatch ticking                    &           &               \\
       Breath                                &  10-20dB  & 60 - 200 mPa  \\
\hline Quiet whisper                         &           &               \\
       Clock ticking                         &  20-30dB  & 200 - 600 mPa \\
\hline Noise indoors                         &  30-40dB  &  0.6 - 2 mPa  \\
\hline Quiet conversation                    &  40-50dB  &   2 - 6 mPa   \\
\hline Normal conversation                   &  50-60dB  &  6 - 20 mPa   \\
\hline Loud conversation                     &  60-70dB  &  20 - 60 mPa  \\
\hline Noisy street                          &  70-80dB  & 60 - 200 mPa  \\
\hline Noise of the truck engine             &   ~80dB   &    200 mPa    \\
\hline Underground railroad noise on the run &           &               \\
       Chisel hammer                         &   ~90dB   &    600 mPa    \\
\hline Loud disco                            & 100-120dB &   2 - 20 Pa   \\
\hline Plane taking-off                      &   120dB   &     20 Pa     \\
\hline Pain threshold                        &   130dB   &     60 Pa     \\
\hline
\end{tabular} \end{center}
\bigskip

Note the perceived pressure range: hearing threshold pressure and pressure produced by plane differ in million times! Therefore, the logarithmic scale is in a better accordance with the ear physiology - the linear change of acoustic pressure does not correspond with sense of loudness linear change. For instance, change of acoustic pressure by 50 mPa during conversation will be extremely sensible; however, it will be absolutely insensible during plane taking-off. But change of acoustic pressure by 6 dB (in twice) will be perceived as an approximately equal loudness change in both cases, though in the first case it will correspond to change of pressure by 25 mPa, and in second case - by 10 Pa.

Another type of loudness is a \definition{record loudness} (\definition{signal loudness}). This loudness is not an acoustic pressure (it can be voltage, magnetic polarization, etc.), however acoustic pressure is produced by acoustic system in accordance with loudness of recorded signal. Each certain signal loudness corresponds to the certain acoustic pressure. Signal loudness can also be measured in decibels. However, while the acoustic pressure is usually measured with respect to hearing threshold (minimum audible acoustic pressure), loudness of digital signal is usually measured with respect to maximum digital level accepted as 0 dB. Thus, loudness of digital signal is represented by negative values (-3 dB, -20 dB) since the record loudness must always be lower than the maximum one. The lower loudness value, the quieter the signal (-20 dB is quieter than -3 dB). Positive loudness of the digital signal means overflow and results in appearance of digital distortions. (These distortions will be considered later).

Volume controls of amplifier, in system settings, of player do not produce any acoustic pressure at all. Without signal even at maximum volume settings, we will hear nothing (with the assumption that the playing system does not produce any noise). Therefore, these settings affect loudness indirectly only and have sense of signal \definition{gain}. (The gain may mean the signal decay too). From here, we will not use the term of loudness for gain levels notation except system loudness since this is established term. We will consider gain level set both in operating system settings and in player, amplifier/receiver, etc. as \definition{system loudness}.

Gain level can also be measured in decibels. It is convenient since in this case the signal loudness and the gain level are simply summed. For instance, signal with loudness of 70 dB gained by 10 dB will sound with loudness of 80 dB. However though loudness and the gain are measured in same units you should distinguish them.

During playing, the recorded signal is converted into acoustic pressure using acoustic system. Let the maximum pressure produced by acoustic system is 100 dB. Then the record with loudness of 0 dB will produce pressure of 100 dB, the record with loudness of - 30 dB will produce pressure of 70 dB, etc. When changing the gain level the absolute loudness will change too. Therefore, when changing the gain level you can always set up a correspondence between the record loudness and the required absolute loudness. For instance, if the conversation level in movie is -30 dB and we want to hear these dialogues with natural loudness, the record loudness must correspond to the pressure of 50 dB. Since when the gain level is maximum (0 dB) the record level of -30 dB produces pressure of 70 dB (too high), while when the level is -20 dB the same record loudness produces the required 50 dB of pressure. Acoustic system adjusted in such way is called \definition{calibrated}, i.e. the calibrated acoustic system is a system with sounds with the correct absolute loudness produced. (In fact, the calibrating process can be much more complex and can include much more parameters, however here and further we will talk about loudness calibrating only). Most of household devices have no labels of the gain level control (or there can be some abstract percents or other meaningless characters), therefore, it is difficult for user to make a precise calibration of acoustic system. 

Hearing \definition{adaptation} also affects loudness perception. Hearing adapts to loudness of surrounding noise and corrects loudness perception accordingly. Certainly, you could see that the man with earphones where loud music is playing begins to talk louder (with point of perception of people around him), however with his point of view it is normal speech loudness (with respect to loudness of earphones sound). Otherwise, in full silence, people begin to whisper and nevertheless it seems to be loud. In conditions of normal city noise, it is difficult to hear ticking of mechanical watches - this sound is perceived as very quiet. But the same watch ticking in full silence at night can be perceived as clear heard sound. Thus, the third type of loudness appears - \definition{subjective loudness}. We will not discuss some kinds of subjective loudness measuring, just talk about qualitative comparing of sounds (louder-quieter).

Thus, sound with the same absolute loudness can be perceived as loud or quiet depending on the environment. Assume that we have adjusted the gain level so as the dialogue loudness in the movie approximately corresponds to reality. So if there is a sound of watch ticking in silence in the movie then we will not hear this sound at all when watching the movie since it is significantly quieter then the surrounding noise and our hearing is adapted to eliminate the surrounding noise. In good hearing conditions when there is no surrounding noise the same watches will be heard clearly.

In most cases, it is convenient to measure record loudness with respect to some reference level. For instance, if the record loudness is -20 dB, is it loud or quiet? And if we know that the dialogue loudness in the same record is -30 dB we can just say that it is loud enough, and if dialogue loudness is -10 dB, we can say that it is quiet enough. \definition{Dialogue level} (average loudness of the conversation) is a very convenient reference level, which can be used for orienting. If sound loudness of 10 dB is higher than the dialogue level then it is loud, and if loudness of 10 dB is lower than the dialogue level then it is quiet. At the same time the dialogue level can be arbitrary and depend on the record - one record may have dialogue level of -10 dB, and the other one may have dialogue level of -30 dB. Anyway, all sounds with loudness which is lower than the dialogue level will be perceived like quiet sounds, while ones with loudness which is higher than the dialogue level will be perceived like the loud sounds, even if the system have been calibrated incorrectly, and dialogues will sound with absolute loudness of 40 dB or 60 dB. Due to the property of adaptation, hearing adapts to the current average loudness and introduces an appropriate 'correction'. However, the correct perception of loudness will be broken when loudness of surrounding noise is approaching dialogue loudness (or even exceeds it). In such a case, all sounds played will seem to be quiet. 

Thus, we can introduce one more loudness scale (in addition to the digital signal loudness scale and the absolute loudness scale) - loudness with respect to the dialogue level. When comparing different loudness we will get the follows:

\bigskip
\begin{footnotesize} \begin{center} \begin{tabularx}{\hsize}{|p{3cm}|X|X|X|X|X|X|}
\hline
  & \bf Loudness of real sounds
  & \bf Loudness with respect to the dialogue level
  & \bf Record-1 level
  & \bf Record-2 level
  & \bf Loudness of record-1 playback
  & \bf Loudness of record-2 playback  \\
\hline \bf Quiet sounds \\
\hline 
  Quiet whisper 
  &  10dB
  &  -40dB
  &  -70dB
  &  -50dB
  &  10dB
  &  10dB \\
\hline
  Noise norm in living quarters
  &  20..30dB
  &  -20..-30dB
  &  -50..-60dB
  &  -30..-40dB
  &  20..30dB
  &  20..30dB \\
\hline 
  Quiet conversation
  &  40dB
  &  -10dB
  &  -40dB
  &  -20dB
  &  40dB
  &  40dB \\
\hline \bf Dialogue level \\
\hline 
  Normal conversation
  &  50dB
  &  0dB
  &  -30dB
  &  -10dB
  &  50dB
  &  50dB \\
\hline \bf Loud sounds \\
\hline
  Noise of a typewriter
  &  70dB
  &  +20dB
  &  -10dB
  &  0dB
  &  70dB
  &  60dB \\
\hline
  Noise of truck engine working
  &  80dB
  &  +30dB
  &  0dB
  &  0dB
  &  80dB
  &  60dB \\
\hline 
  Loud horn within at a distance of 5-7m
  &  100dB
  &  +50dB
  &  0dB
  &  0dB
  &  80dB
  &  60dB \\
\hline
  Noise of a tractor working at a distance of 1m
  &  120dB
  &  +70dB
  &  0dB
  &  0dB
  &  80dB
  &  60dB \\
\hline 
  Pain threshold
  &  130dB
  &  +80dB
  &  0dB
  &  0dB
  &  80dB
  &  60dB \\
\hline
\end{tabularx} \end{center} \end{footnotesize}
\bigskip

There are also two hypothetical records made in different environments given in the table: dialogue level of record-1 is -30 dB while that of record-2 is -10 dB. It is clear that when playing on calibrated system the first record transfers loud sounds much better - it is possible to reproduce sounds with loudness up to 80 dB, while the second record reproduces sounds with loudness up to 60 db only.

Note that two mentioned records require different calibration of acoustic system. Thus, for acoustics that is able to produce pressure of 100 dB, the gain level required for the first record is -20 dB, and that required for the second record is -40 dB. Hence, the first record requires significantly higher gain, and the first record will sound significantly quieter when the gain level settings are equal. Therefore, the second record is suitable for non-calibrated systems since it allows strong deviations of the gain level.

Thus, the record-1 transfers loud sounds well but requires higher gain level; if there is an insufficient gain or some external noise this record will be perceived like extremely quiet. Record-2 does not require high gain, it is heard well even if the gain levels are low or some noise exists, but this record cannot transfer loud sounds well. 

And now let's remember that audio signal is alternating:

\img{pic_loudness/a_signal.png}

What is required for loudness evaluation? Obviously, variation of pressure for one period of acoustic wave has no sense of sound loudness change since we cannot hear single oscillation. Therefore, we must find loudness within the certain time interval but not for one point. There are many different methods for loudness evaluation. The simplest one is to find the maximum and to calculate the signal energy. Methods that are more complex take into account the non-uniformity of hearing of sounds of different frequency and intensity.

When finding loudness as maximum of the signal we scan the range looking for signal maximum: 

\img{pic_loudness/a_loudness.png}

The found level in dB will define loudness. To distinguish this loudness from ones obtained by other methods it is called the \definition{peak level}. From here, we will use only this loudness definition. In fact, this definition does not reflect the real loudness perception well but it is convenient for the next discussion, therefore we will not consider in details any other methods of loudness evaluation even if they are more precise.

%%%%%%%%%%%%%%%%%%%%%%%%%%%%%%%%%%%%%%%%%%%%%%%%%%%%%%%%%%%
% Loudness histogram
%%%%%%%%%%%%%%%%%%%%%%%%%%%%%%%%%%%%%%%%%%%%%%%%%%%%%%%%%%%

\section{Loudness histogram}

\img{pic_loudness/h_fightclub.png}

This picture illustrates loudness distribution of typical DVD movie (DVD FightClub, hereinafter we will consider only original audio tracks with no transformation and no mixing in order not to distort sound pattern). X axis reflects loudness in decibels, Y-axis reflects occurrence rate of sound of this loudness. It is better to say that during the movie playing the current loudness have been continuously controlled. The more often a loudness value occurred during the movie playing the higher the histogram of this part. I.e. we can say that there are few explosions in the movie but at the same time, there are many different background sounds. The histogram is divided into the several conventional areas:

\begin{enumerate}
\item Voice. A dialogue level for movies is one of most important parameters. It is the reference point for all other sounds: sounds with lower level are considered as quiet sounds while the sounds with higher level are considered as loud ones. Good audibility of dialogues is one of the main setting criteria (we will discuss this setting below).
\item Background sounds - the cars going by, footfall sounds, background music
\item Extremely quiet sounds - light breeze, grass rustling, etc.
\item Loud sounds - phone ringing, impact sounds, and so on
\item Explosions and other global crashes
\item Record noise
\end{enumerate}

Boundaries between areas are rather conventional and can be modified for different records. In this case, it is known that the dialogue level is -27 dB.

Looking at the histogram, we can note some interesting moments. Firstly, the most often sounds in the movie were the ones with loudness of -40 dB. Let us just note this fact for now. Secondly, the main hump of histogram is located in area from 0 dB (maximum level) to -70 dB (minimum level). It is reasonable to assume that all significant sounds are inside of this hump while the quieter sounds are just a noise. This value is called \definition{dynamic range} and will be very important further.

This situation is typical enough for DVD movies. Dialogues take much time; background sounds and music take a little more time; some loudness surges occur periodically with the following decays. Generally, sound pattern of this movie is balanced enough and practically full available dynamical range is used.

Let's look at some other movies: 

\img{pic_loudness/h_movies.png}

These examples can be divided into two large groups. First group is characterized by almost full absence of loud sounds (Cube, Dracula, About Schmidt, Savior). Loudness range from 0 dB to -15 dB (very loud sounds) is not practically used here. Otherwise, loud sounds are extremely important for the second group of movies (Interstate60, Blood (Last Vampire), Pitch Black, Matrix). It seems that the number of these sounds is relatively small and they play no special role. However, for instance, the sound of a shot has duration of a split second (the louder part) but exactly high loudness accentuates this sound. If this sound was quieter, there would be an effect of �flatness� and inexpressiveness of the sound. Even when occurring once during the movie loud sound has no great sense in movies of the first type ('quiet' movies); therefore its loudness is not critical so much. Otherwise, in movies of the second type those sounds (even with the small duration) form the mood of the movie.

Another conspicuous feature - high 'humps' in Cube and Interstate 60 movies. There is a constant quiet background hum in the Cube movie. This is why the graph has 'hump' at loudness of approximately -50 dB. The same thing relates to the Interstate 60 movie - it is car travel and, consequently, there is a constant sound of moving car. Shapes of both histograms are very similar; however, the maximum of the histogram for the Interstate 60 movie is lower by 10 dB. As the voice levels are the same in both movies, one can immediately conclude that sound in the Cube movie is much quieter. This conclusion is far from an obvious one. Let us recall hearing adaptation - if we shift graphs to align their maximums at one level, the Cube movie will be perceived as 'quiet' all the same.

The other movies have no such well-marked background sounds, so their graphs are more distributed across the dynamic range.

About Schmidt movie has 'hump' at the level of -70 dB - -80 dB. These are recording noises; and they are relatively loud in this movie, as compared with other movies, but quiet enough and do not hinder viewing.

So far, we considered solely movies. Now let us have a look at music:

\img{pic_loudness/h_music.png}

Right off, it strikes the eye that distribution of loudness is absolutely different. (To show a variety, there are histograms of various music styles having different quality collected. They do not show a real balance of different types of compositions.) All sounds are very much shifted towards the 'loud' part of the dynamic range. The difference between middle-level loudness of movies reaches 40 dB. Besides, a characteristic feature of the movies is a smooth decrease of the histogram in the loud area, whereas for music a histogram maximum often reaches the highest level (0 dB). A small number of such graphs is provided; nevertheless, such situation is very widespread.

The dialogue level concept is often inapplicable to music, thus it is very difficult to separate loud and quiet sounds. Therefore, it is also difficult to determine a subjective loudness of a composition. So, a subjectively loud composition may be objectively quieter than a subjectively loud one. Positive or negative drops of the loudness have a great importance here; these drops are not visible on the histograms.

Absence of a common reference mark results in a chaos - sounds from different sources have different loudness; difference between maximums of composition histograms reaches 20 dB. Besides, in general it is very difficult to formalize a concept of a 'composition's loudness' anyhow. Certainly, many people are familiar with a situation when there is a variety of music from different sources and during its successive playback (e.g., if music is recorded on an audio CD), loudness changes at every passing from one composition to another. It is unpleasant (the mentioned effect of a mismatch between actual loudness and subjective loudness may more embarrass a perception). The provided graphs illustrate this situation very well.

Dynamic range of musical compositions (difference between the loudest and the quietest sounds) is 20-40 dB that is far less than dynamic range (70 dB) of movies. 

%%%%%%%%%%%%%%%%%%%%%%%%%%%%%%%%%%%%%%%%%%%%%%%%%%%%%%%%%%%
% DVD loudness problem
%%%%%%%%%%%%%%%%%%%%%%%%%%%%%%%%%%%%%%%%%%%%%%%%%%%%%%%%%%%

\section{DVD loudness problem}

The average loudness level of musical records (as well as of Windows system sounds) is notably higher than a loudness of DVD movies. Therefore, under the same settings of the system loudness the absolute loudness of movies will be significantly lower. In the presence of external noises loudness may turn out to be insufficient, hearing will be poor (sic!) and it will seem that the sound is of a poor quality. Gain increase in system settings and in amplifier, even to a maximum, may fail to resolve the problem: difference of average loudness reaches 40 dB; it is too high. However, even if an amplifier's power turns out to be sufficient and a DVD is played with sufficient absolute loudness it is not always suitable, as sounds of operating system, which have normal absolute loudness under normal gain settings, will be simply thundering.

This problem is typical mainly for computer playback, as in hardware players gain level is controlled by the decoder itself. Some program DVD players are able to control the system loudness, but it is not always suitable, as in this case loudness of all system sounds will be changed (you may accidentally deafen your neighbors). Besides, in any case, the program cannot control loudness at an external amplifier. So, it is just a partial solution of the problem.

A compromise solution is to process the sound immediately before playback. Such processing can greatly enhance quality of certain record playback under certain conditions. Many people may object that in doing so 'quality' will be lost; however, as it has been already mentioned, there is no absolute quality. We do not set ourselves a goal to cue the sound; our goal is to ensure a pleasant listening to it under our conditions. If an acoustic system does not have sufficient power or if you have nervous neighbors it will be just unpleasant to view a movie with lowered loudness when you cannot distinguish a half of words and a half of quiet sounds is not heard. Even self-suggestion that maximum quality is achieved will not smooth this impression. I should repeat once more the key idea: quality is estimated by our ears. The sound passes many processing stages before it reaches a listener, and a variety of acoustic devices and their characteristics is so wide that the last processing stage before playback is actually a necessity.

%%%%%%%%%%%%%%%%%%%%%%%%%%%%%%%%%%%%%%%%%%%%%%%%%%%%%%%%%%%
% Level change. Overflow, clipping and limiting.
%%%%%%%%%%%%%%%%%%%%%%%%%%%%%%%%%%%%%%%%%%%%%%%%%%%%%%%%%%%

\section{Level change. Overflow, clipping and limiting.}

Level change is simply multiplication of signal amplitude by a defined value. It results in a change (increase or decrease) of the entire signal loudness.

\img{pic_loudness/a_gain.png}

Logarithmically, multiplication by a number is just an addition of a constant. So, if the same number is added to all levels the levels' histogram will be simply shifted:

\img{pic_loudness/h_gain.png}

If we multiply the signal by a too big number, it may result in an overflow. If an overflow is not tracked the amplitude will be equal to virtually random values (see the figure), and this results in frequent clicks, which are very noticeable audibly. The easiest way to struggle with this defect is \definition{clipping}. That is, a signal is clipped when its amplitude goes beyond limits (see the figure). If the overflow is small, the clipping is almost imperceptible audibly; however, when the level increased clipping is heard in the sound as 'sand'.

\img{pic_loudness/a_overflow.png}

More complicated, but more efficient way is \definition{limiting}. It consists in an automatic decrease of the signal level to prevent overflow. To do this, it is necessary to use automatic gain control (AGC), which will adjust t he current signal level. Thus, when there is no overflowing AGS does not change a signal, and as soon as a signal exceeds the maximum level, adjustment is made, which changes the signal to make its level lower than the maximum one. As it can be seen on the figure, the signal completely preserves its form! However, at that loudness remains at maximum.

\img{pic_loudness/l_limiting.png}

It is necessary to take into account that the previous picture shows amplitude and this one shows loudness (defined on the basis of a \emph{large group of amplitudes}). Therefore, notwithstanding that loudness graph is 'cut off', the form of the signal itself is distorted in minimum. This way is not free from shortcomings also. Initially, loudness of the signal itself varied, and after the limitation, all sounds that go beyond the limits have the same loudness. As a result, the sound may become 'flat' and inexpressive. Let us look at histograms (Fight Club):

\img{pic_loudness/h_overflow.png}

An example of unsuccessful gain is shown here. With gain of +10 dB, distortions are still less noticeable (there is still a relatively small number of scenes in the movie where overflow takes place). However, with an increase of gain, the level is limited more and more often, and when the level is equal to +30 dB voice begins to be limited that is highly noticeable. Shortcomings of the limitation are especially noticeable when loudness changes significantly over a short time period - subjectively, loudness begins to jump up and down. If, against a background conversation (which, at the gain of +30 dB, is already reproduced with maximum loudness), loud sound is heard (which is louder than the voice in the original, e.g. a telephone ring) the gain level is decreased to prevent distortion of loud sound, but at that the conversation loudness drops abruptly, too. When the loud sound ceases, the conversation loudness returns to the original level, abruptly as well:

\img{pic_loudness/l_loudness_jump.png}

Thus, relative loudness of various sounds reproduced simultaneously is preserved, but gain level is continuously varying. It is well noticeable and very unpleasant. Thus, at high gain levels (20-30 dB and more) the limitation also produces a bad result.

When gain levels are small, the signal limitation defects are practically unnoticeable, whereas the signal clipping is heard well virtually always. That is, under otherwise equal conditions, limitation gives more good result. The filter always performs either limitation (the \optname{AGC} option is switched on) or clipping (the \optname{AGC} option is switched off). Hereafter, unless otherwise specified explicitly, it will be assumed that the limitation is switched on. Thus, it is recommended to keep the \optname{AGC} option active. 

%%%%%%%%%%%%%%%%%%%%%%%%%%%%%%%%%%%%%%%%%%%%%%%%%%%%%%%%%%%
% Normalization
%%%%%%%%%%%%%%%%%%%%%%%%%%%%%%%%%%%%%%%%%%%%%%%%%%%%%%%%%%%

\section{Normalization}

Let us set a goal to raise loudness without any loss of quality at all. Is it possible? Yes, if there is a 'margin' of the dynamic range. Let us look again at movies' histograms, Cube movie in particular (the yellow graph). One can see on the histogram that loudness does not rise higher than -15 dB (it is possible that in the entire movie there are one-two such occasions; however, for the purposes of this review we will assume that there are no loud sounds at all). So, as there are no loud sounds, it is possible to raise loudness by 15 dB without loss of quality!

\img{pic_loudness/l_normalize.png}

Except for increase of loudness, there are no other changes in the signal. At that, the histogram will shift close by the right-hand margin without changing a form:

\img{pic_loudness/h_normalize.png}

Thus, it will turn out that there is now a sound in the movie having maximum loudness. However, if we continue to increase loudness the distortions (described above) will appear. Such signal form where there is a sound of maximum loudness without introduced distortions is called \definition{normalized} and the process is called \definition{normalization}. Normalization is often carried out at the musical records preparation stage, and the sound reaches us already normalized (e.g. AudioCD). However, the normalization is not carried out for movies. Why?

Let us recall histograms for music and a chaos relating to loudness. For music, it is important to playback with maximum loudness, as it is meant for maximum target audience - CD players, a noisy street, underground, cheap earphones, croaky radio stations where low sounds less than -40 dB are not hearable. (Let us have a one more look at graphs for music - minimum loudness is approximately equal to - 40 dB.)

For DVD it is not so. They are primarily designed for quality home cinema. Playback equipment should be calibrated to reproduce dialogs always with the same absolute loudness for any movie, will it be an action movie with explosions smashing a wall or a melodrama with soft rustle of grass. At that, playback system should always be ready to playback these sounds without necessity to change the gain level manually. Thus, the most important is not a necessity to make a record maximum loud, but a necessity to have fixed reference level to make it possible for a decoder to adjust gain level automatically. This reference level may be any (as a matter of fact, this is not important, i.e. if there is a reference level, corresponding adjustment may be made at any time). The standard de facto for DVD is the dialog level equal to -27 dB. Thus, even if a movie does not contain loud sounds, one should not carry out normalization at the disc creation stage and the dynamic range remains unspent. 

%%%%%%%%%%%%%%%%%%%%%%%%%%%%%%%%%%%%%%%%%%%%%%%%%%%%%%%%%%%
% One-pass Normalization
%%%%%%%%%%%%%%%%%%%%%%%%%%%%%%%%%%%%%%%%%%%%%%%%%%%%%%%%%%%

\section{One-pass normalization}

So, for the purpose of loudness increasing normalization is useful. However, to carry out normalization, one needs to know maximum recording level. To identify it, one needs to review the whole record beforehand. However, it is not always possible and is inconvenient. There is a normalization method, which does not need preview of the whole record - the one-pass normalization. The method consists in a constant search for maximum when we are watching a movie. At the initial moment, the gain is at maximum. When an overflow arises, we decrease gain:

\img{pic_loudness/l_onepass.png}

As it shown on the figure, the first loudness peak is clipped almost in the same way as in the case of usual overflow, but duration of 'clipping' is significantly shorter (compare with simple limiting) and later there will be no overflows. Thus, gain will be adjusted when every new maximum is found, and, as a result, the histogram is automatically shifted to achieve maximum loudness.

As all other, the method is not free from shortcomings. First, the method introduces distortions (though they are low observable, it is necessary to know precisely what to listen to in order to notice the one-pass normalization). Second, loudness is continuously decreasing during a movie show (though usually the main loudness decrease process ends during the first 10-15 minutes of a movie and so it is practically unnoticeable). And third, it may be that the main goal (loudness increase) will not be achieved - everything depends upon the movie itself (it relates also to the usual normalization). Here is an example of histograms obtained in the case of one-pass normalization of the Cube (where normalization is possible) and Fight Club (where normalization will not give the desirable effect) movies: 

\img{pic_loudness/h_onepass.png}

As it seen on these histograms, the result is drastically different. Activation of one-pass normalization gives a substantial positive effect for the Cube movie - sound has become noticeably louder (our assumption that sounds louder than -15 dB are absent in the movie was confirmed - compare with normalization graph). However, in the Fight Club movie no increase of loudness took place at all - thus the desired effect was not achieved, but additional distortions were introduced.

It is necessary to note once more that for one-pass normalization operation preliminary (initial) gain is needed - without level rising, the one-way normalization is senseless. As well as the normal gain, it is set by the \optname{Master} level. However, if it is too strong loudness drop from the beginning of the algorithm operation will be very noticeable. 

%%%%%%%%%%%%%%%%%%%%%%%%%%%%%%%%%%%%%%%%%%%%%%%%%%%%%%%%%%%
% Dynamic range compression
%%%%%%%%%%%%%%%%%%%%%%%%%%%%%%%%%%%%%%%%%%%%%%%%%%%%%%%%%%%

\section{Dynamic range compression}

Let us think: what for do we need to raise loudness? Is it needed in order to hear quiet sounds, which are inaudible in our circumstances (e.g., if it is impossible to listen loud sounds, if there are unwanted noises in the room, etc.)? Is it possible to amplify quiet sounds without processing loud sounds? It turns out that it is possible. This technique is called the Dynamic Range Compression (DRC). For this compression, it is necessary to change the current loudness constantly: to amplify quiet sounds and not to amplify loud ones. The simplest law of loudness change is linear, i.e. loudness is changed according to the law: output\_loudness = k * input\_loudness, where k - dynamic range compression coefficient:

\img{pic_loudness/drc_linear.png}

When k = 1, no changes are made (output loudness is equal to input one). When k < 1, loudness will increase and the dynamic range will be contracted. Let us have a look at the graph (k=1/2) - quiet sound with loudness equal to -50 dB will be higher by 25 dB, that is significantly louder. Meanwhile dialog loudness (-27 dB) will rise only by 13.5 dB, and loudness of the loudest sounds (0 dB) will not change at all. When k > 1, loudness will decrease and the dynamic range will increase.

Let us look at loudness graphs (k = 1/2: the dynamic range will be compressed in double): 

\img{pic_loudness/l_drc.png}

As it may be seen, there were very quiet sounds in the original that were lower than dialog level by 30 dB, as well as very loud ones higher than dialog level by 30 dB. Thus, the dynamic range is 60 dB. After compression, loud sounds are higher just by 15 dB, and quiet ones are lower than dialog level by 15 dB (the dynamic range is now equal to 30 dB). Thus, loud sounds have become significantly quieter, and quiet sounds - significantly louder. At that, there is no overflow!

Let us now consider histograms:

\img{pic_loudness/h_drc.png}

As it clearly seen, when gain level is up to +30 dB the histogram form is well preserved. It means that loud sounds remain well marked (they do not become maximum and are not clipped as in the process of simple gain). At that, quiet sounds are identified. The histogram shows it poorly but the difference is very noticeable by ears. The method's shortcoming: the same jumps of loudness. However, their generation mechanism differs from jumps of loudness arising during clipping, and their character is completely different: they arise generally in the case of very high gain of quiet sounds (and not during clipping of loud sounds as in the process of the normal gain). Excessive compression level results in flattening of the sound pattern: all sounds tend to be equally loud and inexpressive.

High gain of quiet sounds may result in audible recording noise. Thus, the filter uses a somewhat modified algorithm to lessen an increase of noise level: 

\img{pic_loudness/drc_bended.png}

That is, at loudness level equal to -50 dB there is a knee of the transfer function and noises will be amplified less (yellow line). When there is no such knee, noises will be significantly louder (grey line). Such simple modification significantly decreases noise level even at very high compression levels (on the figure - compression is equal to 1.5). The \optname{DRC} level of the filter defines gain level for quiet sounds (-50 dB), thus the compression level 1/5 shown on the figure corresponds to +40 dB level in filter settings.

%%%%%%%%%%%%%%%%%%%%%%%%%%%%%%%%%%%%%%%%%%%%%%%%%%%%%%%%%%%
% Mixing and loudness
%%%%%%%%%%%%%%%%%%%%%%%%%%%%%%%%%%%%%%%%%%%%%%%%%%%%%%%%%%%

\section{Mixing and loudness}

There is one more detail not obvious, which may highly influence loudness during playback of multichannel records on a stereo system (or using headphones). Let us suppose that we have an initial record in the 5.1 format and it is necessary to obtain two channels at the output. Into each output channel there are mixed: one front channel, one rear channel, the LFE channel and part of the central channel:

\begin{verbatim}
L' = L + 0.7*C + SL + LFE
R' = R + 0.7*C + SR + LFE
\end{verbatim}

Let us suppose that there is maximum loud sound simultaneously in all input channels (though it is not probable enough). Then the amplitude in the output channel will be 3.7 times more than maximum (by 11 dB), i.e. there will be the highest overflow. To prevent overflow, mixing formulas should be rewritten in the following way:

\begin{verbatim}
L' = (L + 0.7*C + SL + LFE) / 3.7
R' = (R + 0.7*C + SR + LFE) / 3.7
\end{verbatim}

This is a normalized mixing (you should not confuse it with normalization of record itself) - this guarantees absence of an overflow. However, sound mixed in this way is quieter by 11 dB! Is it possible not to carry out normalization? Yes. However, in this case there are possible overflow and related distortions. Overflow arises only then when all input channels simultaneously reproduce a loud sound. It is not typical for movies (usually rear channels are noticeably quieter than front ones and the LFE channel is used far from always), but it is usual for multichannel musical records. That is why one may leave the non-normalized mixing for musical records, and for music, it is better to switch the normalization on. (In the filter, normalization of mixing is controlled by the \optname{Normalize matrix} option).

\section{Recommendations}

Setting recommendations are highly depend on specific conditions of listening and a goal set. Conditionally, all conditions may be categorized as follows:

\begin{itemize}
\item High-quality home cinema and good listening conditions - no unwanted sounds (street noise, neighbors from above, children running about and a possibility to make noise (neighbors are absent). For this category all post-processing may be switched off (preset standard).
\item Good acoustics, but there is external noise. It is possible to use small gain and small compression level.
\item One cannot make noise (night, neighbors/children), ordinary acoustics (cheap headphones, weak speakers). There is necessary high compression level plus an increase of the level or the one-pass normalization.
\end{itemize}

The best results may be achieved only by combining all methods. In this section, we will consider only a case of viewing DVD movies. For other cases (music, MPEG4 movies, etc.) sound characteristics are very different. However, after familiarization with the filter's setting for DVD-movies it is not difficult to study other cases.

Gain (\optname{Master} level). Jumps of loudness arising in the case of overflow are unnoticeable only then when the sound being clipped is much louder than the basic one: in the example, relating to a telephone ring loudness of the telephone sound is comparable with loudness of voice sound. However, if instead of a telephone there will be a thundering explosion, voice sound will be indistinguishable in any case. Therefore, a limitation of the loudest sounds is admissible and practically unnoticeable. As very loud one may consider sounds from +15...+20 dB as related to dialog level. Thus, when dialog level is equal to -27 dB (de facto for DVD), the admissible gain level is equal to +7...+12 dB. As one more reference, one may address gain histograms: gain up to +10 dB does not highly influence the histogram form, while at gain equal to +20 dB a very large number of sounds will be limited. Thus, one may consider gain up to +10 dB as admissible. This value may be kept practically always: in movies with quiet sound, it will notably raise loudness, and in loud ones, it will not hinder very much. (We should note once more that the reasoning is provided only for the case of watching DVD movies; for music and for most mpeg4 movies it is inapplicable, as histograms' characteristics are very different).

Gain can also be used for limiting signal loudness (e.g. for listening at night). When dialog levels are equal to -27 dB and gain is equal to +17 dB dialog levels after gain will be equal to -10 dB, and level of the loudest sound will be equal, as usual, to 0 dB that is just by +10 dB higher than dialog level. Thus, setting gain in the filter equal to +17 dB and adjusting the system loudness so that dialogs will be reproduced with acceptable absolute loudness, we will get a guarantee that the loudest sounds will not exceed dialog levels by more than 10 dB (though, certainly, in this case signal limitation will be used for loud sounds).

Dynamic range compression (\optname{DRC}). It is much more difficult to determine bounds of using compression, as audibility of compression defects is highly dependable on acoustic system, listening conditions and the listener him(her)self. If histograms are used to solve this issue the upper limit of compression applicability may be estimated as +20...+30 dB (see histograms). In this case, loud sounds still remain distinguished from quiet ones by loudness. Thus, compression level is selected only audibly, until the necessary loudness is achieved and defects will remain unnoticeable.

It is necessary to take into account that compression and gain are applied simultaneously. Thus, when the gain level is +20 dB and compression is twofold (+25 dB) the actual gain level will be equal to +10 dB. It is normal, as a requirement for gain is also decreased with an increase of compression.

One-pass normalization (\optname{One-pass norm} option). It is also applicable practically always. For high-quality systems with low gain level it will allow to diminish number of overflows; and in other cases, combined with compression, it will allow to achieve maximum loudness, also with minimum of distortions related to overflow. An initial gain level (\optname{Master} level) for normalization is selected on the basis of goals; if a high gain is not needed the desirable gain is set. To achieve maximum loudness, one may set a level equal to +20 dB. 

\end{document}
